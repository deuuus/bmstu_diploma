\begin{essay}{}
    \noindent\textbf{Ключевые слова}: цифровое изображение, преобразование Фурье, искажение, дефокусировка, размытие, функция рассеяния точки, конволюция, <<слепая>> деконволюция, спектр, кепстр.

    Объектом исследования является цифровое изображение, искаженное дефокусировкой фотокамеры.
    
    Целью работы являлась разработка метода восстановления дефокусированных изображений на основе определенных параметров искажения. 
    
    Для достижения поставленной цели были решены следующие задачи:
    
    \begin{itemize}
    	\item[---] проведен анализ предметной области восстановления дефокусированных изображений;
    	\item[---] проведен сравнительный анализ методов восстановления без учета априорной информации об искажении;
    	\item[---] разработан метод восстановления дефокусированных изображений на основе определенных параметров искажения;
    	\item[---] спроектировано и реализовано программное обеспечение для реализации разрабатываемого метода;
    	\item[---] разработанный метод исследован на применимость при работе с различными типами дефокусированных изображений. 
    \end{itemize}
    
    Разработанный метод имеет ряд ограничений, однако может быть применим даже в условиях отсутствия априорной информации об искажающем процессе.
    
    Было выявлено, что зависимость времени обработки изображения от его размера и цветовой модели имеет линейный вид, а обработка цветного изображения в среднем требует примерно в 3 раза больше времени, чем обработка серого. 
    
    Была определена область применимости предложенного метода: в диапазоне радиуса дефокусировки от 10 до 20 единиц точность восстановления является удовлетворительный, а в случае обработки серого изображения этот диапазон еще шире, поэтому если учет информации о цвете изображения некритичен, то рекомендуется использовать серое изображение вместо цветного в целях сокращения времени обработки и повышения качества результата.
    
    В качестве перспектив дальнейшего развития реализованного программного обеспечения можно рассмотреть учет сложных моделей дефокусировки, возникающих в реальных условиях, применимость к видеофайлам, а также адаптацию предложенного решения под другие языки программирования в целях достижения независимости и кроссплатформенности.
\end{essay}