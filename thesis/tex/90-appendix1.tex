\begin{appendices}
\chapter{Полный код функции слепой деконволюции}

В листингах А.1 - А.3 представлен полный код функции слепой деконволюции.

\begin{lstlisting}[caption={Функция слепой деконволюции}]
function focused_img = my_blind_deconvolution(original_img)
	import cepstrum.*
	import radius.*
	import psf.*
	
	function img = do_gray()
		img_cepstrum = cepstrum(original_img);
		focus_radius = radius(img_cepstrum);
		focus_psf = psf(focus_radius);
		med_img = deconvlucy(original_img, focus_psf, 100);
		
		if focus_radius - 1 > 0
			focus_radius = focus_radius - 1;
			focus_psf = psf(focus_radius);
		end
		sub_img = deconvlucy(original_img, focus_psf, 100);
		
		focus_radius = focus_radius + 2;
		focus_psf = psf(focus_radius);
		add_img = deconvlucy(original_img, focus_psf, 100);
		
		psnr_value_med = psnr(med_img, original_img);
		psnr_value_sub = psnr(sub_img, original_img);
		psnr_value_add = psnr(add_img, original_img);
		
		[~, max_index] = max([psnr_value_med psnr_value_sub psnr_value_add]);
		
		if max_index == 1
			img = med_img;
		elseif max_index == 2
			img = sub_img;
		else
			img = add_img;
		end
	end	
\end{lstlisting}
\clearpage
\begin{lstlisting}[caption={Функция слепой деконволюции (продолжение)}]
	function channel_radius = get_radius(channel)
		channel_cepstrum = cepstrum(channel);
		channel_radius = radius(channel_cepstrum);
	end
 
	function focused_channel = apply_psf(channel, radius)
		chanel_focus_psf = psf(radius);
		focused_channel = deconvlucy(channel, chanel_focus_psf, 100);
	end
	
	function img = deconvolve_with_radiuses(red_channel_radius, green_channel_radius, blue_channel_radius)
		red_channel_focused = apply_psf(red_channel, red_channel_radius);
		green_channel_focused = apply_psf(green_channel, green_channel_radius);
		blue_channel_focused = apply_psf(blue_channel, blue_channel_radius);
		img = cat(3, red_channel_focused, green_channel_focused, blue_channel_focused);
	end
	
	function update_radiuses_neg()
		if (red_channel_radius - 1) ~= 0
			red_channel_radius = red_channel_radius - 1;
		end
		if (green_channel_radius - 1) ~= 0
			green_channel_radius = green_channel_radius - 1;
		end
		if (blue_channel_radius - 1) ~= 0
			blue_channel_radius = blue_channel_radius - 1;
		end
	end
	
	function update_radiuses_pos()
		red_channel_radius = red_channel_radius + 2;
		green_channel_radius = green_channel_radius + 2;
		blue_channel_radius = blue_channel_radius + 2;
	end
	
	if length(size(original_img)) == 2
		focused_img = do_gray();
	else
		red_channel = original_img(:, :, 1);
		green_channel = original_img(:, :, 2);
		blue_channel = original_img(:, :, 3);		
\end{lstlisting}
\clearpage
\begin{lstlisting}[caption={Функция слепой деконволюции (продолжение)}]
		red_channel_radius = get_radius(red_channel);
		green_channel_radius = get_radius(green_channel);
		blue_channel_radius = get_radius(blue_channel);
		
		med_rgb_img = deconvolve_with_radiuses(red_channel_radius, green_channel_radius, blue_channel_radius);
		
		update_radiuses_neg();
		sub_rgb_img = deconvolve_with_radiuses(red_channel_radius, green_channel_radius, blue_channel_radius);
		
		update_radiuses_pos();
		add_rgb_img = deconvolve_with_radiuses(red_channel_radius, green_channel_radius, blue_channel_radius);
		
		rgb_psnr_value_med = psnr(med_rgb_img, original_img);
		rgb_psnr_value_sub = psnr(sub_rgb_img, original_img);
		rgb_psnr_value_add = psnr(add_rgb_img, original_img);
		
		[~, max_index] = max([rgb_psnr_value_med rgb_psnr_value_sub rgb_psnr_value_add]);
		
		if max_index == 1
			focused_img = med_rgb_img;
		elseif max_index == 2
			focused_img = sub_rgb_img;
		else
			focused_img = add_rgb_img;
		end
	end
end
\end{lstlisting}

\end{appendices}