\chapter*{ВВЕДЕНИЕ}
\addcontentsline{toc}{chapter}{ВВЕДЕНИЕ}

Среди способов восприятия человеком информации об окружающем мире посредством органов чувств зрение занимает особое место --- с помощью глаз в среднем воспринимается до 80~\% информации~\cite{percent}, поступающей из внешней среды. Именно поэтому зрительные образы, часто запечатляемые снимками фотокамеры, играют важнейшую роль в жизни человека. 

Системы фотосъемки могут быть использованы в различных сферах деятельности: криминалистика, кинематография, микроэлектроника, биомедицина, археология, исследования космоса, оборонное производство~\cite{usage}.

Однако часто полученный снимок фотокамеры оказывается искаженным за счет различных причин: шумы, турбулентность атмосферы, элементы интерференции и размытие, вызванное дефокусировкой, движением и нелинейностью пленки. Установлено, что дефокусировка и смаз --- наиболее частые дефекты при съемке~\cite{degradate} .

Повторное получение изображений в указанных ранее сферах с целью устранения искажений часто является либо дорогостоящим, либо вовсе невозможным.

Большинство искажений являются частично или полностью обратимыми, таким образом, несмотря на распространенное заблуждение, потеря информации в исследуемом изображении может быть устранена. Например, дефокусировка является частично обратимым преобразованием: качество восстановления существенно зависит от наличия априорной информации о процессе искажения. 