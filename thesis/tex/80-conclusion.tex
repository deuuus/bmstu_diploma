\chapter*{ЗАКЛЮЧЕНИЕ}
\addcontentsline{toc}{chapter}{ЗАКЛЮЧЕНИЕ} 

В данной работе была рассмотрена задача восстановления дефокусированных изображений на основе определенных параметров искажения.

Для достижения поставленной задачи был проведен анализ предметной области дефокусированных изображений, а также существующих методов восстановления без учета априорной информации об искажении. Было выяснено, что функция рассеяния точки в случае дефокусировки фотокамеры имеет форму диска. 

Для решения задачи <<слепой>> деконволюции был предложен метод, согласно которому ФРТ, описывающая искажающий процесс, определялась на основе радиуса колец, которые наблюдаются на структуре кепстра дефокусированных изображений. Также были рассмотрены критерии оценки качества восстановления, основным из которых является пиковое соотношение сигнал~--~шум.

В результате выполнения данной работы было спроектировано и реализовано соответствующее программное обеспечение, позволяющее пользователю загружать искаженное изображение, применять восстановление к нему и сохранять результат. Таким образом, цель работы достигнута.

Разработанный метод имеет ряд ограничений, однако может быть применим даже в условиях отсутствия априорной информации об искажающем процессе. Согласно исследованиям, область применимости предложенного метода~---~изображения с радиусом дефокусировки от 10 до 20 пикселей. Также было выяснено, что в случае, когда пренебержение цветовой информацией является некритичным, то рекомендуется обрабатывать изображение в тонах серого. %Алгоритм не применим для абсолютно всех размытий и подходит лишь под определенный класс задач.

В качестве перспектив дальнейшего развития можно рассмотреть несколько направлений, таких как учет сложной модели искажения, возникающего в реальных ситуациях помимо моделирования известных ФРТ; улучшение обработки шума на изображении; применение <<слепой>> деконволюции к видеофайлам; развитие комбинированных методов деконволюции, охватывающих более широкую область применения; учет частичной дефокусировки.