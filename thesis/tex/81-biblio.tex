\addcontentsline{toc}{chapter}{СПИСОК ИСПОЛЬЗОВАННЫХ ИСТОЧНИКОВ}
\renewcommand\bibname{СПИСОК ИСПОЛЬЗОВАННЫХ ИСТОЧНИКОВ}
\bibliographystyle{utf8gost705u}  % стилевой файл для оформления по ГОСТу
\begin{thebibliography}{3}
    \makeatletter
    \def\@biblabel#1{#1. }

    \bibitem{percent}
    Раков Д. Л. Морфологический анализ и синтез невозможных объектов //Невозможные объекты и оптические иллюзии в современном искусстве и дизайне (Традиционные и компьютерные технологии). – 2014. – С. 16-19.
    
    \bibitem{usage}
    Констандогло А. В. Методы восстановления расфокусированных и смазанных изображений //Политехнический молодежный журнал. – 2020. – №. 5. – С. 5-5.

    \bibitem{degradate}
    Малыхина Г. Ф., Меркушева А. В. Сеть с симметричной функцией преобразования нейронов для подавления искажений и восстановления изображения //Научное приборостроение. – 2008. – Т. 18. – №. 2. – С. 73-80.

    \bibitem{def1}
    Фотоаппараты и съемочные фотографические объективы //Государтсвенный стандарт союза ССР. - 1983.

    \bibitem{db_percent}
    Бадюк А. А. АНАЛИЗ ИННОВАЦИОННОГО РОСТА НА РЫНКЕЦИФРОВЫХ ФОТОАППАРАТОВ //Экономика и социум. – 2019. – №. 12 (67). – С. 236-242.
    
    \bibitem{obj}
    Фомин А. В. Общий курс фотографии //М.:«Лёгкая индустрия. – 1975.
    
    \bibitem{digital_pic}
    Гонсалес Р., Вудс Р. Цифровая обработка изображений. – Litres, 2022.
    
    \bibitem{digit_pic}
    Гришенцев А. Ю., Коробейников А. Г. Методы и модели цифровой обработки изображений. – 2014.
    
    \bibitem{scheme}
    Шемплинер В. В. Восстановление дефокусированных изображений методом двумерного преобразования Фурье и регуляризации Тихонова //Научно-технический вестник информационных технологий, механики и оптики. – 2008. – №. 48. – С. 56-66.

    \bibitem{diskk}
    Чочича, П. А. Определение вида и параметров искажений изображения по Фурье-спектру сигнала //Институт проблем передачи информации. - 2019.

    \bibitem{spectrr}
    Бойко Б. П. и др. Спектр сигнала.

    \bibitem{flowers}
    A bunch of pink flowers that are blooming [Электронный ресурс]. Режим доступа: https://unsplash.com/photos/OBLADaipjv4 (дата обращения 24.05.2023).
    
    \bibitem{cepstrr}
    Соловьев В. Е. Неэталонная оценка качества в задачах фильтрации, восстановления и сжатия изображений : дис. – Владимирский государственный университет имени Александра Григорьевича и Николая Григорьевича Столетовых, 2013.

    \bibitem{ceps}
    Ахмад Х. М. Сравнительное исследование эффективности различных методов кепстрального описания речевых сигналов в задачах распознавания //Вестник Тамбовского государственного технического университета. – 2007. – Т. 13. – №. 4. – С. 887-891.

    \bibitem{rings}
    Dosselmann R. W., Yang X. D. No-reference noise and blur detection via the fourier transform //Dept. of Computer Science, Univ. of Regina. Regina, SK (Saskatchewan), Canada. – 2012.

    \bibitem{svertka}
    Воропаева Н. В. и др. Дискретное преобразование Фурье в обработке сигналов //Самара: Изд-во" Самар. ун. – 2015. – Т. 2015.

    \bibitem{conv}
    Convolution [Электронный ресурс]. Режим доступа: https://paperswithcode.com/method/convolution (дата обращения 22.05.2023).

    \bibitem{equ}
    Набиев М., Дорогиницкий М. М. Восстановление размытых изображений //Математическое моделирование и информационные технологии. – 2018. – С. 34-35.

    \bibitem{inv_noise}
    Сизиков В. С., Лавров А. В. Устойчивые методы математико-компьютерной обработки изображений и спектров //СПб: Университет ИТМО. – 2018.

    \bibitem{winer}
    Чурсин М. А. ВОССТАНОВЛЕНИЕ РАСФОКУСИРОВАННЫХ И СМАЗАННЫХ ИЗОБРАЖЕНИЙ //Россия молодая: передовые технологии – в промышленность!. – 2013. – №. 2. – С. 099-101.
    
    \bibitem{blind_def}
    Переславцева Е. Е., Филиппов М. В. Метод ускоренного восстановления изображений, смазанных при движении //Машиностроение и компьютерные технологии. – 2012. – №. 02. – С. 30.

    \bibitem{psnr_ssim}
    Hore A., Ziou D. Image quality metrics: PSNR vs. SSIM //2010 20th international conference on pattern recognition. – IEEE, 2010. – С. 2366-2369.

    \bibitem{matlab_doc}
    Math. Graphics. Programming. [Электронный ресурс]. Режим доступа: https://www.mathworks.com/products/matlab.html (дата обращения 22.05.2023).

    \bibitem{matlab}
    Лазарев Ю. Ф. Начала программирования в среде MatLAB: Учебное пособие //К.: НТУУ «КПИ. – 2003.

    \bibitem{image_toolbox}
    Image Processing Toolbox [Электронный ресурс]. Режим доступа: https://www.mathworks.com/products/image.html (дата обращения 22.05.2023).
    
    \bibitem{signal_toolbox}
    Signal Processing Toolbox [Электронный ресурс]. Режим доступа: https://www.mathworks.com/products/signal.html (дата обращения 22.05.2023).

    \bibitem{compiler}
    MATLAB Compiler [Электронный ресурс]. Режим доступа: https://www.mathworks.com/products/compiler.html (дата обращения 22.05.2023).

    \bibitem{runtime}
    MATLAB Runtime [Электронный ресурс]. Режим доступа: https://www.mathworks.com/products/compiler/matlab-runtime.html (дата обращения 22.05.2023).

    \bibitem{car}
    Car Number [Электронный ресурс]. Режим доступа: https://mavink.com/post/260AB0E02ED9872BD33ED4F3575DEDF72CAM (дата обращения 24.05.2023).

    \bibitem{cat}
    Cats meow to say hello [Электронный ресурс]. Режим доступа: https://www.worldsbestcatlitter.com/wp-content/uploads/2017/05/1-hello.jpg (дата обращения 24.05.2023).
    
\end{thebibliography}
