\chapter*{ЗАКЛЮЧЕНИЕ}
\addcontentsline{toc}{chapter}{ЗАКЛЮЧЕНИЕ} 

В результате выполнения научно-исследовательской работы была достигнута поставленная цель: были изученены и классифицированы известные методы восстановления изображений, искаженных дефокусировкой фотокамеры.

Для достижения поставленной цели были выполнены следующие задачи: 

\begin{itemize}
	\item проведен анализ предметной области восстановления изображений, искаженных дефокусировкой фотокамеры;
	\item приведен обзор существующих методов восстановления изображений, искаженных дефокусировкой фотокамеры;
	\item сформулированы критерии сравнения рассмотренных методов;
	\item классифицированы рассмотренные методы;
	\item на основе полученных теоретических сведений сделаны выводы.
\end{itemize}

Были рассмотрены следующие методы восстановления дефокусированных изображений: инверсный фильтр, фильтр Винера, метод Люси~--~Ричардсона, метод <<слепой>> деконволюции, регуляризация Тихонова. 

В результате классификации были выделены два класса по следующим критериям: по наличию информации об искажающей функции и по области обработки.
 
Очевидно, что методы деконволюции при известной искажающей функции работают лучше, чем <<слепые>> методы, однако на практике ядро искажающей функции часто является неизвестным, поэтому применение слепых методов более актуально в реальных задачах. Выбор области обработки (пространственная или частотная) цифрового изображения зависит от требований и допущений конкретной решаемой задачи.

В качестве перспектив развития поставленной задачи можно рассмотреть разработку и программную реализацию метода восстановления дефокусированных изображений на основе определенных параметров искажения.