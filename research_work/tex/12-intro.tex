\chapter*{ВВЕДЕНИЕ}
\addcontentsline{toc}{chapter}{ВВЕДЕНИЕ}

Среди способов восприятия человеком информации об окружающем мире посредством органов чувств зрение занимает особое место --- с помощью глаз в среднем воспринимается до 80~\% информации, поступающей из внешней среды.~\cite{percent} Именно поэтому зрительные образы, часто запечатляемые снимками фотокамеры, играют важнейшую роль в нашей жизни. 

Системы фотосъемки могут быть использованы в различных сферах деятельности: криминалистика, кинематография, микроэлектроника, биомедицина, археология, исследования космоса, оборонное производство.~\cite{spheres}

Однако часто полученный снимок фотокамеры оказывается искаженным за счет различных причин: шумы, элементы интерференции и размытие, вызванное дефокусировкой, движением и нелинейностью пленки.~\cite{damage}

Целью данной научно-исследовательской работы является изучение и классификация известных методов восстановления изображений, искаженных дефокусировкой фотокамеры.

Для достижения поставленной цели необходимо выполнить следующие задачи:

\begin{itemize}
	\item провести анализ предметной области восстановления изображений, искаженных дефокусировкой фотокамеры;
	\item провести обзор существующих методов восстановления изображений, искаженных дефокусировкой фотокамеры;
	\item сформулировать критерии сравнения рассмотренных методов;
	\item классифицировать рассмотренные методы;
	\item на основе полученных теоретических сведений сделать выводы.
\end{itemize}