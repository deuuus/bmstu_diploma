\begin{essay}{}
    \noindent\textbf{Ключевые слова}: восстановление искаженных изображений, дефокусировка фотокамеры, функция размытия точки, операция свертки, теорема о свертке, Фурье~--~преобразования, классификация.\\
    
    Объектом исследования является изображение, искаженное дефокусировкой фотокамеры.
    
    В результате был проведен краткий обзор следующий методов восстановления дефокусированных изображений: инверсная фильтрация, фильтр Винера, регуляризация Тихонова, метод Люси~--~Ричардсона, метод <<слепой>> деконволюции. 
    
    В результате классификации были выделены два класса методов по следующим критериям: по наличию информации об искажающей функции и по области вычислений.
\end{essay}