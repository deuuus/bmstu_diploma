\chapter{Обзор известных методов восстановления изображений, искаженных дефокусировкой фотокамеры}
\label{cha:design}

\section{Инверсная фильтрация}

Одним из самых простых методов решения задачи восстановления изображения, искаженного дефокусировкой фотокамеры, является инверсная фильтрация. Использование данного метода предполагает, что искажающая функция заранее известна.

Согласно теореме о свертке, имеем (2.1):

\begin{equation}
	G(u,\;v) = H(u,\;v) \cdot F(u,\;v) + N(u,\;v).
\end{equation}

Было предложено разделить обе части выражения на $H(u,\;v)$ и получить следующую оценку исходного изображения:

\begin{equation}
	\hat{F}(u,\;v) = \frac{G(u,\;v)}{H(u,\;v)} = F(u,\;v) + \frac{N(u,\;v)}{H(u,\;v)}.
\end{equation}

Если на изображении отсутствует шум, то восстановление происходит достаточно точно. Однако, если шум присутствует, то составляющая $\cfrac{N(u,\;v)}{H(u,\;v)}$ стремится к бесконечности в виду того, что в частотной области функция $H(u,\;v)$ стремится к нулю, что приводит к получению некачественного результата.

\section{Фильтр Винера}

Усовершенствованием инверсной фильтрации можно считать фильтр Винера.~\cite{noise_reasons} Данный метод, в отличие от предыдущего, учитывает информацию о шуме на изображении.

Метод базируется на рассмотрении функций изображения и шума как случайных процессов и нахождении такой оценки $\hat{F}$ для неискаженного изображения $f$, чтобы среднеквадратическое отклонение этих величин было минимальным~\cite{viner}:

\begin{equation}
    \hat{F}(u,\;v) = \frac{1}{H(u,\;v)}\cdot \frac{|H(u,\;v)|^2}{|H(u,\;v)|^2 + \cfrac{S_\eta(u,\;v)}{S_f(u,\;v)}} \cdot G(u,\;v).
\end{equation}

Функцией $S$ обозначают энергетические спектры шума и исходного изображения соответственно. Т.~к. эти величины обычно неизвестны, то их заменяют на некоторую константу $K$, которую можно охарактеризовать как приблизительное соотношение сигнал-шум:

\begin{equation}
    \hat{F}(u,\;v) = \frac{1}{H(u,\;v)}\cdot \frac{|H(u,\;v)|^2}{|H(u,\;v)|^2 + K} \cdot G(u,\;v).
\end{equation}

Если шум на изображении отсутствует, то фильтр Винера сводится к инверсной фильтрации. Основной недостаток фильтра Винера заключается в наличии краевых эффектов, проявляющихся в виде осциллирующей помехи (ряби или полос).

\section{Регуляризация Тихонова}

Метод также реализовывается в частотной области и предполагает наличие информации об искажающей функции. Этот метод также называют методом минимизации сглаживающего функционала со связью, или методом наименьших квадратов со связью.

Идея заключается в формулировке задачи в матричном виде с дальнейшим решением соответствующей задачи оптимизации.~\cite{tihon} Решение задачи имеет следующий вид:

\begin{equation}
	\hat{F}(u,\;v) = \frac{H'(u,\;v)}{|H(u,\;v)|^2 + \gamma|P(u,\;v)|^2} \cdot G(u,\;v),
\end{equation}

где $\gamma$ --- параметр регуляризации, $P(u,\;v)$ --- результат Фурье~--~преобразования оператора Лапласа, $H'(u,\;v)$ --- функция, комплексно сопряженная $H(u,\;v)$.

Если параметр $\gamma = 0$, то данный метод сводится к инверсной фильтрации.

\section{Метод Люси~--~Ричардсона}

Этот метод является итерационным, предполагает наличие информации об искажающей функции и, в отличие от предыдущих методов, реализовывается в пространственной области и является нелинейным.~\cite{lusi} 

Идея заключается в использовании метода максимального правдоподобия, для которого предполагается, что изображение подчиняется распределению Пуассона.% (этот закон также называют <<законом редких событий>>).

Математические выражение для данного метода имеет следующий вид:

\begin{equation}
	\hat{f}_{k+1}(x,\;y) = \hat{f}_k(x,\;y)\cdot \left( h(-x,\;-y) \oplus \frac{g(x,\;y)}{h(x,\;y) \oplus \hat{f}_k(x,\;y)} \right),
\end{equation}

где $\hat{f}_{k+1}$ --- оценка изображения $f$ на $k$-ом шаге вычислений. 

Недостатком этого метода являются краевые эффекты в виде горизонтальных и вертикальных волос на изображении. Также возникает вопрос выбора критерия остановки итерационного алгоритма.

\section{Метод <<слепой>> деконволюции}

Название данного метода отражает тот факт, что искажающая функция заранее неизвестна, и этот метод относится к другому классу алгоритмов в отличие от ранее рассмотренных. Слепая деконволюция является некорректной задачей --- количество неизвестных параметров превышает количество известных.

Как правило, метод состоит из двух этапов обработки:

\begin{enumerate}
    \item Ядро размытия оценивается по входному изображению.
    \item Используя оценочное ядро, применяется стандартный алгоритм деконволюции для оценки скрытого изображения.
\end{enumerate}

Слепая деконволюция основывается на методе максимального правдоподобия, где целевой функцией является исходное (неискаженное) изображение.

\section*{Вывод}

Из рассмотренных выше алгоритмов только инверсный фильтр не учитывает шумовую составляющую, что делает его непригодным для реальных задач повышения качества изображений. Наибольшую вычислительную сложность имеют итерационные алгоритмы: метод Люси~--~Ричардсона и метод <<слепой>> деконволюции. Таким образом, можно выделить два класса алгоритмов: по наличию информации об искажающей функции и по области вычислений.